\chap{Introduction}

The \lrstar parser generator system is composed of two programs ---
\lrstare \& \dfae --- that generate a table-driven syntax and lexcial
parsers respectively.  A runtime library is that is also included that
is used; it is used to interpret the data in the generated tables and
parse the input source files.

Many grammars have been provided to help showcase the expressive style
of the EBNF grammars accepted by \lrstar.  They can be used as the
starting point of your new compiler project.

Additionally, several example projects are included to showcase how
specific parsing problems can be met by using lrstar.  To name a few:

\begin{itemize}
\item Context-sensitive symbol interpretation

  In the \texttt{C} programming language, determining if a
  \emph{symbol} is a \texttt{typedef} requires more information than
  is present with a text-only parse of an input file.  With a properly
  written grammar, \lrstar system can automatically promote a symbol
  to a \emph{semantic symbol} so that parse tree will properly match
  the source, without additional grammar work.


\item Multiple parsers in a single program

  Sometimes it's desirable to have a single program parse multiple
  languages.  This examples shows that it is easy to do with the
  \lrstar system.


\item LR(\emph{k}) parsing

  This example shows that a generated LR(\emph{k}) parser is simple to
  with proper invocation of the \lrstar tooling.

\item JSON
  This implements and small parser for JSON, with the result being an
  appropriate internal data structure for any valid JSON input.
\end{itemize}


This reference will show how  syntactic and lexcial grammars are
created, and how to use the \lrstar system to generate a working
parser for them.  But, more than that, it will show how to use the
\lrstar runtime system to fill out the parser into a tool that produce
useful work.

The syntactic grammars are specified using \tbnfgrmfn, while the lexical
grammars are specified via \dfagrmfn.  A user of this software must
understand both of these systems able to effectively read and write
grammars.
